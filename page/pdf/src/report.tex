\documentclass[10pt]{article}

\usepackage[a4paper, left=2cm, right=2cm, bottom=2cm]{geometry} % A4 paper size and thin margins

\usepackage{graphicx}
\graphicspath{ {../imgs/} }

\usepackage{blindtext}
\usepackage{everypage}
\usepackage{environ}
\newcounter{abspage}

\usepackage{fancyhdr}
\pagestyle{fancy}
\fancyhf{}
\rhead{Climate Smart Pilots}
\lhead{NSW Department of Primary Industries}
\cfoot{This data is released by NSW Department of Primary Industries for evaluation purposes only and should not be relied upon for business decisions, carrying out risk assessments or other uses.}

\usepackage{float}

\usepackage{indentfirst}
\usepackage[colorlinks=true,linkcolor=black,anchorcolor=black,citecolor=black,filecolor=black,menucolor=black,runcolor=black,urlcolor=blue]{hyperref}
\usepackage{xcolor}

\usepackage[utf8]{inputenc}
\usepackage[T1]{fontenc}
\usepackage[sfdefault]{ClearSans}

\begin{document}


\begin{titlepage}
		\parbox[t]{0.93\textwidth}{
			\parbox[t]{0.91\textwidth}{
				\raggedright
				\fontsize{50pt}{80pt}\selectfont
				\vspace{0.7cm}
				\includegraphics[width=0.5\textwidth]{DPI_logo.png}
				
				Clyde River Water\\
				Quality Report\\
				\vspace{0.7cm}
			}
		}
	\vfill
	\parbox[t]{0.93\textwidth}{ 
		\raggedleft
		\large
		{\Large Climate Smart Pilots}\\[4pt] 
		\today \\
		\vspace{0.5cm}
		\texttt{www.farmdecisiontech.net.au}\\
		
		\hfill\rule{0.2\linewidth}{1pt}
	}
\end{titlepage}

\section*{Foreword}

\subsection*{Funding}
This work has been produced by the NSW Primary Industries Climate Change Research Strategy funded by the NSW Climate Change Fund.

\subsection*{NSW Department of Primary Industries Disclaimer}
This is a research trial and pilot project, and you should not rely solely on the information or advice provided in these reports.

\subsection*{Feedback and Questions}
Please provide feedback and questions to Harvey Bates \\ \\
Email: \href{mailto:harvey.bates@dpi.nsw.gov.au}{harvey.bates@dpi.nsw.gov.au} \\ \\
Ph: 0447 359 557

\pagebreak

\tableofcontents

\listoffigures

\pagebreak

\section{Salinity}
\subsection{Weekly}

\begin{figure}[H]
\centering
\includegraphics[width=\textwidth]{weekly-salinity-extremes.png}
\caption[Weekly Minimum and Maximum Salinity]{These values represent the highest and lowest salinity reading a buoy has recorded in each of the harvest areas in the past week.}
\end{figure}

\begin{figure}[H]
\centering
\includegraphics[width=\textwidth]{weekly-salinity-chart.png}
\caption[Average Weekly Salinity Chart]{This figure represents the daily average salinity of all buoys contained within a harvest area. Its designed to reduce the impact of tides and provide the general trend of salinity over the past week.}
\end{figure}

\begin{figure}[H]
\centering
\includegraphics[width=\textwidth]{weekly-salinity.png}
\caption[Average Weekly Salinity Table]{This figure represents the daily average salinity of each of the buoys contained within a harvest area. Its designed to reduce the impact of tides and provide the general trend of salinity over the past week for specific locations within harvest areas.}
\end{figure}

\pagebreak

\subsection{Fortnightly}

\begin{figure}[H]
\centering
\includegraphics[width=\textwidth]{fortnightly-salinity.png}
\caption[Average Fortnightly Salinity Difference]{This figure demonstrates the average difference in salinity for this week compared with the prior week. It displays a longer term, general trend to see if salinity is increasing, stabilising or decreasing.}
\end{figure}

\pagebreak

\newpage
\section{Water Temperature}
\subsection{Weekly}

\begin{figure}[H]
\centering
\includegraphics[width=\textwidth]{weekly-temperature-extremes.png}
\caption[Weekly Minimum and Maximum Temperature]{These values represent the highest and lowest temperature reading a buoy has recorded in each of the harvest areas in the past week.}
\end{figure}

\begin{figure}[H]
\centering
\includegraphics[width=\textwidth]{weekly-temperature-chart.png}
\caption[Average Weekly Temperature Chart]{This figure represents the daily average temperature of all buoys contained within a harvest area. Its designed to reduce the impact of tides and provide the general trend of temperature over the past week.}
\end{figure}

\begin{figure}[H]
\centering
\includegraphics[width=\textwidth]{weekly-temperature.png}
\caption[Average Weekly Temperature Table]{This figure represents the daily average temperature of each of the buoys contained within a harvest area. Its designed to reduce the impact of tides and provide the general trend of temperature over the past week for specific locations within harvest areas.}
\end{figure}

\subsection{Fortnightly}

\begin{figure}[H]
\centering
\includegraphics[width=\textwidth]{fortnightly-temperature.png}
\caption[Average Fortnightly Temperature Difference]{This figure demonstrates the average difference in temperature for this week compared with the prior week. It displays a longer term, general trend to see if temperature is increasing, stabilising or decreasing.}
\end{figure}

\section{Precipitation}
\subsection{Weekly}

\begin{figure}[H]
\centering
\includegraphics[width=\textwidth]{weekly-precipitation.png}
\caption[Daily Total Precipitation Budd Island]{Daily total precipitation at Budd Island for the past week.}
\end{figure}

\subsection{Yearly}
\begin{figure}[H]
\centering
\includegraphics[width=\textwidth]{yearly-precipitation.png}
\caption[Yearly Cumulative Precipitation Budd Island]{Compares this years total precipitation against previous years.}
\end{figure}

\section{Flow Rate from Tributaries}
\subsection{Fortnightly}

\begin{figure}[H]
\centering
\includegraphics[width=\textwidth]{fortnightly-brooman.png}
\caption[Fortnightly Discharge From Brooman Tributary]{Fortnightly (daily total) water discharge from Brooman.}
\end{figure}

\begin{figure}[H]
\centering
\includegraphics[width=\textwidth]{fortnightly-buckenbowra.png}
\caption[Fortnightly Discharge Rate Buckenbowra]{Fortnightly (daily total) water discharge from Buckenbowra.}
\end{figure}

\subsection{Yearly}
\begin{figure}[H]
\centering
\includegraphics[width=\textwidth]{combined-dischargerate.png}
\caption[Yearly Cumulative Discharge Rate Brooman]{Compares this years total water discharge against previous years. Drought years (2018 and 2019) are shown for comparison.}
\end{figure}

\end{document}








